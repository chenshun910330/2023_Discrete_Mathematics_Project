%%%%%%%%%%%%%%%%%%%%%%%%%%%%%%%%%%%%%%%%%%%%%%%%%%
%%%%%%%%%%%%%%%%%%%%%%%%%%%%%%%%%%%%%%%%%%%%%%%%%%
%%
%% Based on the "beamer-greek-two" template provided 
%% by the Laboratory of Computational Mathematics, 
%% Mathematical Software and Digital Typography, 
%% Department of Mathematics, University of the Aegean
%% (http://myria.math.aegean.gr/labs/dt/)
%%
%% Adapted by John Liaperdos, October-November 2014
%% (ioannis.liaperdos@gmail.com)
%%
%% Last update: 22/06/2017 (English Support)
%%


%%%%%%%%%%%%%%%%%%%%%%%%%%%%%%%%%%%%%%%%%%%%%%%%%%
%%%%%%%%%%%%%%%%%%%%%%%%%%%%%%%%%%%%%%%%%%%%%%%%%%
%%
\PassOptionsToPackage{unicode}{hyperref}
\PassOptionsToPackage{naturalnames}{hyperref}
\documentclass[utf8]{beamer} 
%\usepackage{babel}
%\usepackage[utf8]{inputenc}


%%% FONT SELECTION %%%%%%%%%%%%%%%%%
%%% we choose a sans font %%%%%%%%%%
\usepackage{kmath,kerkis} 
%\usepackage[default]{gfsneohellenic} 
%%%%%%%%%%%%%%%%%%%%%%%%%%%%%%%%%%%%

\usepackage{color}
\usepackage{amsmath}
\usepackage{amssymb}

\usepackage{epstopdf}
\usepackage{graphicx}
\usepackage{xeCJK}  %%李:設定中文
\setCJKmainfont{Noto Serif CJK SC}  %%李:設定字體
\setCJKmonofont{Noto Serif CJK SC}
\XeTeXlinebreaklocale "zh" %文字間隔

\XeTeXlinebreakskip = 0pt plus 1pt
\graphicspath{{./images/}}

%%
% load TEI-Pel - specific layout
\usepackage{TeiPel_En_Beamer_Layout}  %李:這裡要用setting的文檔名,不要去改名字
\setTeipelLayout{draft,newlogo}% options: "draft", "newlogo"



%%%%%%%%%%%%%%%%%%%%%%%%%%%%%%%%%%%%%%%%%%%%%%%%%%%%%%%%%%%%
% Thesis Info %%%%%%%%%%%%%%%%%%%%%%%%%%%%%%%%%%%%%%%%%%%%%%
%%%%%%%%%%%%%%%%%%%%%%%%%%%%%%%%%%%%%%%%%%%%%%%%%%%%%%%%%%%%
	% title
		\title[遞移關係計數]{有限集合上遞移關係數量的計數問題:從暴力法到邊際效能優化}
	% author 
	% author 
    % (In the mandatory argument "{}", separate multiple
    % authors with "\and" - use "\\" for better author name formatting
    % in the title page. In the optional argument "[]" include all
	% author names, with no "\and" or text formatting macros.)
	% Example: 
    \author[A. Author 林辰訓]{林辰訓}
		%\author[A. Author]{Anthony Author}
	% supervisor	
		\supervisor{指導教授}{姚為成}{教授}
	% date
		\presentationDate{June 1, 2023}

  %李:設定行距
  \linespread{1.25}
%%%%%%%%%%%%%%%%

\begin{document}

% typeset front slides
	\typesetFrontSlides






%%%%%%%%%%%%%%%%
% Your Slides Start here:

%%%%%%%%%%%%%%%%第一部分
\section{簡介}

\begin{frame}{大綱}
\small
本研究探討關係矩陣之遞移性檢測演算法,並計算數量為 $n$ 之集合上的遞移關係總數。
\begin{itemize}
    \item \textbf{製作動機}:為什麼會選這個題目
    \item \textbf{基礎理論}:以 $n=2$ 為例,列舉 $2^{n^2}$ 種關係並檢驗遞移性。
    \item \textbf{演算法實作}:從基礎暴力法 ($O(n^3)$) 到優化演算法。
    \item \textbf{效能瓶頸分析}:探討組合爆炸問題(Combinatorial Explosion),解釋為何當 $n > 5$ 時運算量激增。
    \item \textbf{實驗結果}:比對優化前後之加速倍率,並對照 OEIS A006905 序列數值。
\end{itemize}
\end{frame}


%%第一部分第一小節
%第一頁
\begin{frame}{基礎定義:遞移律 (Transitivity)}
    \begin{block}{數學定義}
        設 $R$ 為集合 $A$ 上的一個關係。若對於所有的 $a, b, c \in A$:
        \[ (a, b) \in R \quad \text{且} \quad (b, c) \in R \implies (a, c) \in R \]
        則稱 $R$ 具有\textbf{遞移性}。
    \end{block}

    \vspace{10pt}
    \begin{itemize}
        \item \textbf{直觀理解}:如果有一條路徑可以從 $a$ 走到 $b$,再從 $b$ 走到 $c$,那麼必須存在一條「直達」的邊從 $a$ 直接連向 $c$。
        \item \textbf{矩陣觀點}:在關係矩陣中,這代表若 $M_{ab}=1$ 且 $M_{bc}=1$,則 $M_{ac}$ 必須為 $1$。
    \end{itemize}
\end{frame}



%% 第一部分第二小節:製作動機
%%%%%%%%%%%%%%%%%%%%%%%%%%%%%%%%%%%%%%%%%%%%%%%%%%
% 第二頁
\begin{frame}{製作動機}
    \begin{exampleblock}{從理論到實作的探索}
        \small
        \begin{itemize}
            \item<1-> \textbf{驗證理論定義}:
            課本上的遞移律定義較抽象,我想透過程式窮舉所有可能,觀察遞移關係在集合中的真實分佈。
            
            \item<2-> \textbf{感受規模成長}:
            當集合元素增加時,關係總數以 $2^{n^2}$ 劇烈成長。我想親自實作,體驗這種「組合爆炸」對計算帶來的挑戰。
            
            \item<3-> \textbf{尋求運算效率}:
            在面對數以億計的矩陣時,單純的暴力法是否足以應付?我想嘗試不同的程式邏輯來提升檢測速度。
        \end{itemize}
    \end{exampleblock}
\end{frame}


%%%%%%%%%%%%%%%%第二部分
\section{實作}


%%第二部分第一小節
\begin{frame}{實作前思考:實例分析 ($n=2$)}
    \footnotesize % 全局設定為較小字體
    針對集合 $A=\{1, 2\}$,其關係總數為 $2^{n^2}=16$ 種。
    
    \setlength{\tabcolsep}{3pt} % 縮減表格欄位間距
    \begin{table}
        \centering
        \begin{tabular}{l|l|l}
            \hline
            \textbf{序對數量} & \textbf{代表性組合} & \textbf{遞移性檢查} \\ \hline
            0 組 (1項) & $\emptyset$ & 滿足 (Trivial) \\
            1 組 (4項) & $\{(1,1)\}, \{(1,2)\}, \dots$ & 滿足 \\
            2 組 (6項) & $\{(1,1),(1,2)\}, \dots$ & \textcolor{red}{$\{(1,2),(2,1)\} \to$ 缺 $(1,1)$} \\
            3 組 (4項) & $\{(1,1),(1,2),(2,2)\}, \dots$ & \textcolor{red}{$\{(1,1),(1,2),(2,1)\} \to$ 缺 $(2,2)$} \\
            4 組 (1項) & $\{(1,1),(1,2),(2,1),(2,2)\}$ & 滿足 \\ \hline
        \end{tabular}
    \end{table}

    \begin{columns}[T]
        \begin{column}{0.5\textwidth}
            \begin{alertblock}{\scriptsize 非遞移反例判定}
                \scriptsize
                \begin{itemize}
                    \item $\{(1,2), (2,1)\} \implies$ 缺少 $(1,1)$
                    \item $\{(1,2), (2,1), (2,2)\} \implies$ 缺少 $(1,1)$
                    \item $\{(1,1), (1,2), (2,1)\} \implies$ 缺少 $(2,2)$
                \end{itemize}
            \end{alertblock}
        \end{column}
        \begin{column}{0.45\textwidth}
            \vspace{5pt}
            \textbf{統計結論:} \\
            總關係數:$16$ \\
            非遞移項:$3$ \\
            \textbf{遞移關係總數:$\mathbf{13}$}
        \end{column}
    \end{columns}
\end{frame}

\begin{frame}{實作前思考:從集合到矩陣}
    \small
    \begin{block}{轉化動機}
        集合枚舉隨著 $n$ 增加會面臨災難。改用 \textbf{矩陣} 表示關係 $R$,可將遞移性判斷轉化為矩陣運算。
    \end{block}

    \begin{itemize}
        \item \textbf{矩陣表示}:$a_{ij} = 1 \iff (i, j) \in R$。
        \item \textbf{遞移判定}:若 $a_{ik} = 1$ 且 $a_{kj} = 1$,則必須 $a_{ij} = 1$。
    \end{itemize}

    \begin{columns}[c]
        \begin{column}{0.45\textwidth}
            \centering
            滿足遞移性範例 \\
            $\{(1,1),(1,2)\}$ \\
            $\begin{bmatrix} 1 & 1 \\ 0 & 0 \end{bmatrix}$
        \end{column}
        \begin{column}{0.1\textwidth}
            \centering \Large vs
        \end{column}
        \begin{column}{0.45\textwidth}
            \centering
            \textcolor{red}{不具}遞移性範例 \\
            $\{(1,2),(2,1),(2,2)\}$ \\
            $\begin{bmatrix} 0 & 1 \\ 1 & 1 \end{bmatrix}$
        \end{column}
    \end{columns}
\end{frame}
%%第二部分第二小節

%第九頁
\subsection{遇到的瓶頸}

\begin{frame}{遇到的瓶頸:窮舉法的困境}
    \scriptsize % 整頁使用極小字體
    為了確保不遺漏任何遞移關係,我嘗試「暴力枚舉」所有可能的關係矩陣。
    
    \begin{columns}[T] % 開始分欄,T代表頂端對齊
        \begin{column}{0.48\textwidth}
            \begin{block}{枚舉邏輯}
                \begin{itemize}
                    \item 一個 $n \times n$ 矩陣共有 $n^2$ 格。
                    \item 每格有 $0, 1$ 兩種可能,總組合為 $2^{n^2}$。
                    \item 使用整數 $i$ 的位元依序填入矩陣。
                \end{itemize}
            \end{block}
            
            \begin{exampleblock}{實例:$n=2, i=6$}
                $i=6 \implies (0110)_2$
                \[
                \begin{bmatrix} 
                a_{00}=0 & a_{01}=1 \\ 
                a_{10}=1 & a_{11}=0 
                \end{bmatrix}
                \]
                \textbf{判定}:存在 $(0,1)$ 與 $(1,0)$,但 $a_{00}=0$(缺少 $(0,0)$),故為非遞移關係。
            \end{exampleblock}
        \end{column}

        \begin{column}{0.48\textwidth}
            \begin{alertblock}{效能瓶頸:組合爆炸}
                隨著 $n$ 增加,計算量呈指數成長:
                \begin{itemize}
                    \item \textbf{$n=4$}:65,536 次 $\to$ 瞬間完成。
                    \item \textbf{$n=5$}:3,355 萬次 $\to$ 約 1 秒。
                    \item \textbf{$n=6$}:\textbf{687 億次} $\to$ \textbf{計算量激增 2048 倍}。
                \end{itemize}
            \end{alertblock}
            
            \vspace{5pt}
            \textbf{結論}:\\
            單純的「產生矩陣 $\to$ 檢查」邏輯在 $n=6$ 時會撞上硬體性能牆。
        \end{column}
    \end{columns}
\end{frame}

\subsection{演算法}

\begin{frame}[fragile]{演算法 A:基礎暴力法 (Naive)}
    \footnotesize
    這是最直觀的定義實作,使用三層嵌套迴圈檢查所有三元組 $(i, k, j)$。
    
    \begin{block}{Algorithm: is\_transitive\_naive(Matrix $M$, $n$)}
        \begin{itemize}
            \item \textbf{For} $i = 0$ \textbf{to} $n-1$:
            \item \quad \textbf{For} $k = 0$ \textbf{to} $n-1$:
            \item \quad \quad \textbf{If} $M[i][k] == 1$: \hfill \textit{\color{gray}// 若存在 $i \to k$}
            \item \quad \quad \quad \textbf{For} $j = 0$ \textbf{to} $n-1$:
            \item \quad \quad \quad \quad \textbf{If} $M[k][j] == 1$ \textbf{and} $M[i][j] == 0$:
            \item \quad \quad \quad \quad \quad \textbf{Return False} \hfill \textit{\color{red}// 違反遞移律}
        \end{itemize}
        \textbf{Return True}
    \end{block}

    \begin{itemize}
        \item \textbf{複雜度}:$O(n^3)$
        \item \textbf{缺點}:內層迴圈無法利用現代處理器的並行特性。
    \end{itemize}
\end{frame}

\begin{frame}[fragile]{演算法 B:位元並行優化法 (Bit-vector)}
    \footnotesize
    利用電腦字長一次處理整列的特性,將 $O(n)$ 的內層迴圈壓平。
    
    \begin{block}{Algorithm: is\_transitive\_opt(Rows $R$, $n$)}
        \begin{itemize}
            \item \textbf{For} $i = 0$ \textbf{to} $n-1$:
            \item \quad \textbf{For} $k = 0$ \textbf{to} $n-1$:
            \item \quad \quad \textbf{If} ($k$-th bit of $R[i]$ is $1$): \hfill \textit{\color{gray}// 若存在 $i \to k$}
            \item \quad \quad \quad \textit{\color{teal}// 檢查:Row[k] 是否為 Row[i] 的子集}
            \item \quad \quad \quad \textbf{If} ($R[k]$ \textbf{AND} $R[i]$) $\neq R[k]$:
            \item \quad \quad \quad \quad \textbf{Return False}
        \end{itemize}
        \textbf{Return True}
    \end{block}

    \begin{itemize}
        \item \textbf{複雜度}:實質降至 $O(n^2)$ (當 $n \le 64$)
        \item \textbf{關鍵運算}:將 $n$ 次判斷簡化為單一位元運算指令。
    \end{itemize}
\end{frame}




\begin{frame}{優化原理}
    \scriptsize
    \begin{columns}[T]
        % 左半部:範例與定義
        \begin{column}{0.5\textwidth}
            \textbf{1. 關係矩陣的位元向量化}
            \begin{itemize}
                \item \textbf{Row 0}: $0\ 1\ 0$ ($0 \to 1$) $\to$ \texttt{dec: 2}
                \item \textbf{Row 1}: $0\ 0\ 1$ ($1 \to 2$) $\to$ \texttt{dec: 4}
                \item \textbf{Row 2}: $0\ 0\ 0$ (無連向) $\to$ \texttt{dec: 0}
            \end{itemize}
            
            \vspace{5pt}
            \textbf{2. 判定邏輯 ($i=0, k=1$)}
            \begin{itemize}
                \item 既然 $0 \to 1$,則 $1$ 能到的地方,$0$ 也必須能到。
                \item \textbf{條件}:Row 1 必須是 Row 0 的「子集」。
                \item \textbf{運算}:\texttt{(R[1] \& R[0]) == R[1]}
            \end{itemize}
        \end{column}

        % 右半部:運算細節與效能(往上擠,避開下方)
        \begin{column}{0.45\textwidth}
            \begin{exampleblock}{運算過程實測}
                \begin{itemize}
                    \item \texttt{R[1]: 0 0 1}
                    \item \texttt{R[0]: 0 1 0}
                    \item \texttt{AND} 結果:\texttt{0 0 0}
                \end{itemize}
                \textbf{結果}:\texttt{000} $\neq$ \texttt{001} (判定失敗) \\
                $\implies$ 此關係非遞移。
            \end{exampleblock}

            \begin{block}{優化效益}
                \begin{itemize}
                    \item \textbf{暴力法}:需用迴圈逐一檢查每個 $j$,耗時 $O(n)$。
                    \item \textbf{位元優化}:利用 CPU \texttt{AND} 指令一次處理整列。速度提升約 $n$ 倍。
                \end{itemize}
            \end{block}
            
            % 這裡留白,給右下角的 Logo 空間
            \vspace{3cm} 
        \end{column}
    \end{columns}
\end{frame}


\begin{frame}{位元優化實測:以 $i=51$ 為例}
    \footnotesize
    $i=51$ 的二進位映射為 \texttt{000 110 011},對應矩陣:
    \begin{center}
        Row 0: \texttt{000} \quad Row 1: \texttt{110} \quad Row 2: \texttt{011}
    \end{center}
    
    根據程式邏輯,系統會依序檢查矩陣中值為 1 的位置:
    
    \begin{block}{驗證步驟 (追蹤矩陣中的 1)}
        \begin{enumerate}
            \item \textbf{驗證 (1,0)}:
            1 可以走到 0,需滿足 \texttt{Row 0 \& Row 1 == Row 0}。 \\
            運算:\texttt{000 \& 110 = 000} \hfill (成立 \checkmark)
            
            \item \textbf{驗證 (1,1)}:
            1 可以走到 1,需滿足 \texttt{Row 1 \& Row 1 == Row 1}。 \\
            運算:\texttt{110 \& 110 = 110} \hfill (成立 \checkmark)
            
            \item \textbf{驗證 (2,1)}:
            2 可以走到 1,需滿足 \texttt{Row 1 \& Row 2 == Row 1}。 \\
            運算:\texttt{110 \& 011 = 010} \\
            結果:\texttt{010} $\neq$ \texttt{Row 1 (110)} \hfill (\textbf{不成立 \scalebox{0.8}{X}})
        \end{enumerate}
    \end{block}

    

    \textbf{結論}:
    由於驗證 (2,1) 失敗,代表存在路徑 $2 \to 1 \to 2$ 但缺少直接關係 $2 \to 2$。程式觸發 \textbf{Early Exit},判定此關係非遞移。
\end{frame}


\section{成果與討論}
\subsection{遭遇困難}

%第九頁
\begin{frame}{遭遇困難}
    \scriptsize
    \begin{itemize}
        \item \textbf{運算量爆炸性成長}:
        當 $n=5$ 時,C 語言不到 1 秒就跑完 3,355 萬個組合;但 $n=6$ 直接飆升到 \textbf{687 億}個。即便 C 語言效率很高,這種「千倍等級」的差距還是讓普通電腦跑得非常吃力,這是我第一次親身體會到演算法複雜度的威力。
        
        \item \textbf{找不到規律走捷徑}:
        我原本想嘗試用遞迴的方法,看能不能用 $n-1$ 的結果推導出 $n$ 的數量。但後來發現「遞移性」只要矩陣中任一個位子改變,整個性質就得重新判定,沒辦法「偷懶」少算,只能老老實實地全部窮舉。

        \item \textbf{卡在語法細節與 Debug}:
        在實作過程中,我花了很多時間在處理模組瑣碎的語法。原本以為邏輯對了就沒事,但常常因為括號位置、位元運算優先權,或是陣列索引值算錯,導致跑出來的答案跟正確數列完全對不起來。為了找出這些小 bug,耗費了大量的精神去反覆檢查。
    \end{itemize}
\end{frame}

\begin{frame}{數據對照:OEIS A006905 數列}
    \small
    我將程式跑出來的結果與網路上 OEIS 數列對照,確認前 5 項完全正確:
    \begin{center}
        \begin{tabular}{|c|r|r|}
            \hline
            $n$ & 總矩陣數量 ($2^{n^2}$) & 遞移關係數量 (我的結果) \\
            \hline
            1 & 2 & 2 \\
            2 & 16 & 13 \\
            3 & 512 & 171 \\
            4 & 65,536 & 3,994 \\
            5 & 33,554,432 & 154,303 \\
            \hline
            \textbf{6} & \textbf{687 億} & \textbf{等待運算中...} \\
            \hline
        \end{tabular}
    \end{center}
    \footnotesize 註:$n=6$ 的計算量是 $n=5$ 的 2048 倍,這就是為什麼電腦會卡住的原因。
\end{frame}

\begin{frame}{數據對照:OEIS A006905 數列}
    \begin{figure}[h]
    \includegraphics[totalheight=6.5cm]{8.PNG}
    \end{figure}
\end{frame}

\subsection{心得}

\section{成果}

\begin{frame}{學到的技巧}
    \small
    在寫程式的過程中,我學到了幾個關鍵技巧:
    \begin{itemize}
        \item \textbf{位元運算的魔力}:
        以前課本說的 \texttt{<<} (左移) 和 \texttt{\&} (位元與) 真的很實用。我用一個整數來代表矩陣的一列,這樣判斷遞移性時,速度比寫三層迴圈快非常多。
        
        \item \textbf{及時判斷 (Early Exit)}:
        不用把整個矩陣檢查完。只要發現有一個地方不符合 $a_{ik}=1, a_{kj}=1 \implies a_{ij}=1$,就立刻跳出,這樣可以節省很多時間。

        \item \textbf{狀態空間的映射 (State Mapping)}:
        學習如何將一個超大的整數 $i$ 透過位元偏移(Offset)映射回 $n \times n$ 的關係矩陣。這讓我也理解到,當狀態數達到 $2^{n^2}$ 時,如何高效地在「整數索引」與「矩陣結構」間轉換是處理組合問題的核心。
    \end{itemize}
\end{frame}

\subsection{照片}

\begin{frame}{照片}
\framesubtitle{compiler}
    \begin{figure}[h]
    \includegraphics[totalheight=6.5cm]{20.PNG}
    \end{figure}
\end{frame}


%%%%
\subsection{參考資料}
\begin{frame}{參考資料}
    \begin{thebibliography}{9}
        \bibitem{g2} 所有資料(latex開發碼/c code/照片) [網址] \url{https://github.com/david910330d1d1d1d110-cyber/561}
        \bibitem{g2} 前17項遞移關係數量、相關證明 [網址] \url{https://www.researchgate.net/publication/352383148_On_the_number_of_transitive_relations_on_a_set}
        \bibitem{g3} 其他關係的補充 [網址] \url{https://www.youtube.com/watch?v=GvNGf9Gki7o}
        \bibitem{g1} C 的時間模組. [網址] \url{https://pydoing.blogspot.com/2010/07/c-stdtime.html}

    \end{thebibliography}
\end{frame}





%%
\end{document}

%動畫示範1
%\begin{frame}{Slide Title \#2}
 %\framesubtitle{Slide subtitle \#1}
	%\begin{itemize}
		%\item Use the \texttt{itemize} environment frequently.
		%\pause
		%\item Use short sentences and phrases.
		%\pause
		%\item In this presentation we use the \textbackslash{}\texttt{pause} macro.
	%\end{itemize}
 %\end{frame}
%動畫示範2:非順序提示
 \begin{frame}{Slide Title \#2}
	\begin{itemize}
		\item <1->You can define the order of appearance.
		\item <3->Like here.
		\item <2->This is the second item to appear.
	\end{itemize}
\end{frame}

%block example
\begin{frame}{Slide Title \#4}
	\begin{example}
		<1->First example. 
	\end{example}
	\begin{example}
		<2->Second example.
	\end{example}
\end{frame}


%table example
\begin{frame}{Slide Title \#5}
	\begin{center}
		Table example \\[12pt]
		\begin{tabular}{c||c|c|c|}
			& \textbf{col 1} & \textbf{col  2} & \textbf{col 3} \\
			\hline
			\hline
			\textbf{row 1} & 11 & 12 & 13 \\
			\hline
			\textbf{row 2} & 21 & 22 & 23 \\
		\end{tabular}
    \end{center}
\end{frame}

\begin{frame}{Slide Title \#6}
	\begin{center}
		Figure example \\[12pt]
		\includegraphics[width=0.35\textwidth,keepaspectratio]{LampFlowchart.png}
		\\
		\footnotesize(source: \textlatin{Wikipedia})
    \end{center}
\end{frame}

\begin{frame}{Slide Title \#7}
	\centering
	Math examples \\[12pt]
	\begin{equation}
        	B'=-\nabla \times E
	\end{equation}
	\begin{equation*}
        	E'=\nabla \times B - 4\pi j
	\end{equation*}
\end{frame}

%alert box
\begin{frame}{Summary}
   	\begin{alertblock}{Attention}
   		\textlatin{This is an important alert}
   	\end{alertblock}
\end{frame}


%references
\begin{frame}{References}
	\begin{thebibliography}{2}
		\beamertemplatebookbibitems
		\bibitem{Author1990}A.\ Author. \newblock\emph{Handbook of Everything}.\newblock
\textlatin{Some Press, \oldstylenums{1990}}.

		\beamertemplatearticlebibitems
		\bibitem{Someone2002}B.\ Author.\newblock On this and that\emph{.}
\newblock\emph{Journal on This and That}. 
\oldstylenums{2}(\oldstylenums{1}):\oldstylenums{50}--\oldstylenums{100}, 
\oldstylenums{2000}.
	\end{thebibliography}
\end{frame}
