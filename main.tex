%%%%%%%%%%%%%%%%%%%%%%%%%%%%%%%%%%%%%%%%%%%%%%%%%%
%%%%%%%%%%%%%%%%%%%%%%%%%%%%%%%%%%%%%%%%%%%%%%%%%%
%%
%% Based on the "beamer-greek-two" template provided 
%% by the Laboratory of Computational Mathematics, 
%% Mathematical Software and Digital Typography, 
%% Department of Mathematics, University of the Aegean
%% (http://myria.math.aegean.gr/labs/dt/)
%%
%% Adapted by John Liaperdos, October-November 2014
%% (ioannis.liaperdos@gmail.com)
%%
%% Last update: 22/06/2017 (English Support)
%%


%%%%%%%%%%%%%%%%%%%%%%%%%%%%%%%%%%%%%%%%%%%%%%%%%%
%%%%%%%%%%%%%%%%%%%%%%%%%%%%%%%%%%%%%%%%%%%%%%%%%%
%%
\PassOptionsToPackage{unicode}{hyperref}
\PassOptionsToPackage{naturalnames}{hyperref}
\documentclass[utf8]{beamer} 
%\usepackage{babel}
%\usepackage[utf8]{inputenc}


%%% FONT SELECTION %%%%%%%%%%%%%%%%%
%%% we choose a sans font %%%%%%%%%%
\usepackage{kmath,kerkis} 
%\usepackage[default]{gfsneohellenic} 
%%%%%%%%%%%%%%%%%%%%%%%%%%%%%%%%%%%%

\usepackage{color}
\usepackage{amsmath}
\usepackage{amssymb}

\usepackage{epstopdf}
\usepackage{graphicx}
\usepackage{xeCJK}  %%李:設定中文
\setCJKmainfont{Noto Serif CJK SC}  %%李:設定字體
\setCJKmonofont{Noto Serif CJK SC}
\XeTeXlinebreaklocale "zh" %文字間隔

\XeTeXlinebreakskip = 0pt plus 1pt
\graphicspath{{./images/}}

%%
% load TEI-Pel - specific layout
\usepackage{TeiPel_En_Beamer_Layout}  %李:這裡要用setting的文檔名,不要去改名字
\setTeipelLayout{draft,newlogo}% options: "draft", "newlogo"



%%%%%%%%%%%%%%%%%%%%%%%%%%%%%%%%%%%%%%%%%%%%%%%%%%%%%%%%%%%%
% Thesis Info %%%%%%%%%%%%%%%%%%%%%%%%%%%%%%%%%%%%%%%%%%%%%%
%%%%%%%%%%%%%%%%%%%%%%%%%%%%%%%%%%%%%%%%%%%%%%%%%%%%%%%%%%%%
	% title
		\title[離散期末報告]{找出數量為n的集合上有多少遞移關係的數量,對所有正整數 $n$。給出正整數 $n$ 的一般結果(不僅對 $n \leq 7$)。}
	% author 
	% author 
    % (In the mandatory argument "{}", separate multiple
    % authors with "\and" - use "\\" for better author name formatting
    % in the title page. In the optional argument "[]" include all
	% author names, with no "\and" or text formatting macros.)
	% Example: 
    \author[A. Author 林辰訓]{林辰訓}
		%\author[A. Author]{Anthony Author}
	% supervisor	
		\supervisor{指導教授}{姚為成}{教授}
	% date
		\presentationDate{June 1, 2023}

  %李:設定行距
  \linespread{1.25}
%%%%%%%%%%%%%%%%

\begin{document}

% typeset front slides
	\typesetFrontSlides






%%%%%%%%%%%%%%%%
% Your Slides Start here:

%%%%%%%%%%%%%%%%第一部分
\section{簡介}


%%第一部分第一小節
\subsection{須知}
%第一頁
\begin{frame}{須知}
\small
\begin{flushleft}
\textbf{遞移性關係(Transitive Relations)}:是一種關係,其特點是如果元素 $A$ 和元素 $B$ 之間存在這種關係,且元素 $B$ 和元素 $C$ 之間也存在這種關係,則元素 $A$ 和元素 $C$ 之間必須存在這種關係。
\\\textbf{遞移關係定義}:$\forall a, b, c \in A$,若 $(a, b) \in R$ 且 $(b, c) \in R$,則 $(a, c) \in R$。
\end{flushleft}
\end{frame}



\subsection{大綱}
%第一頁
\begin{frame}{大綱}
\small
\begin{flushleft}
探討關係矩陣的遞移性概念以及如何檢測具有遞移關係的矩陣。主要內容如下:

\begin{itemize}
\item 探討集合內只有兩個元素的情況(n=2),並列出所有可能的組合方式,找到具有遞移關係的組合。
\item 利用定義,跑出所有矩陣中的元素 (i, k) 和 (k, j),檢查是否存在 (i, j) 來判斷矩陣的遞移性。
\item 使用 Python 程式碼生成 n*n 的矩陣,並使用位元運算和迴圈生成所有可能的矩陣。檢查所有矩陣,判斷是否具有遞移關係。
\item 印出具有遞移關係的矩陣,並計算具有遞移關係的矩陣的數量。
\end{itemize}

總結:雖然找到了具有遞移關係的矩陣並計算出了數量。但是因為時間複雜度過高,只能計算出前幾個元素的遞移關係數量。
\end{flushleft}
\end{frame}




%%第一部分第二小節
\subsection{製作動機}

%第二頁
\begin{frame}{製作動機}
    \begin{exampleblock}  %灰綠格子 
		<1->{加強集合的概念與程式碼的實作}
  %標題與格子出現順序編碼
  %放大會放不下
  {\linespread{1}\small
    \selectfont
		}
  {\linespread{1}\small
    \selectfont
        \textbf{\begin{itemize}
            \item  深化對數學邏輯、集合論和關係代數等領域的理解。
            \item  在圖論、數據分析和人工智慧等領域很常使用
            \item  探討NP hard的問題
        \end{itemize}}}
        \end{exampleblock}

\end{frame}


%%%%%%%%%%%%%%%%第二部分
\section{製作全過程}


%%第二部分第一小節
\subsection{製作前思考}
%第五頁
\begin{frame}{製作前思考}
	\begin{exampleblock}
	    <1->{假設集合裡有2個元素1和2,(i.e. A=\{1,2\})}
     {\linespread{1}\footnotesize
    \selectfont
        集合$A \times A$=\{(1,1),(1,2),(2,1),(2,2)\}它有$2^{n^2} = 16$種組合方式,我們要在這些可能裡面找到具有遞移關係的集合,因為n=2而已,所以我先全部列出來方便大家觀察}
	\end{exampleblock}
		
\end{frame}

\begin{frame}{製作前思考}
    \begin{itemize}
        \item \textcolor{teal}{\textbf{挑選}}
        \\\small\selectfont{    當我挑選0組pair時:\{\}
    \\
    當我挑選1組pair時:\{(1,1)\}、\{(1,2)\}、\{(2,1)\}、\{(2,2)\}
    \\
    當我挑選2組pair時:\{(1,1),(1,2)\}、\{(1,1),(2,1)\}、\{(1,1),(2,2)\}、
    \par~~~~~~~~~~~~~~~~~~~~~~~~~~~~~\{(1,2),(2,1)\}、\{(1,2),(2,2)\}、\{(2,1),(2,2)\}
    \\
    當我挑選3組pair時:\{(1,1),(1,2),(2,1)\}、\{(1,1),(1,2),(2,2)\}、
    \par~~~~~~~~~~~~~~~~~~~~~~~~~~~~~\{(1,1),(2,1),(2,2)\}、\{(1,2),(2,1),(2,2)\}
    \\
    當我挑選4組pair時:\{(1,1),(1,2),(2,1),(2,2)\}}
        \item \textcolor{teal}{\textbf{判斷哪些不具遞移關係}}
        \\\small\selectfont{\textbf{\{(1,2),(2,1)\}不具有遞移關係},因為(1,1)不在集合裡面;
    \\
    \textbf{\{(1,1),(1,2),(2,1)\}不具有遞移關係},因為(2,2)不在集合裡面;
    \\
    \textbf{\{(1,2),(2,1),(2,2)\}不具有遞移關係},因為(1,1)不在集合裡面。
    \\
    因此當n=2時,總共有16種可能,有13種關係具有遞移性。}
    \end{itemize}
\end{frame}

\begin{frame}{製作前思考}
    \begin{itemize}
        \small
        \item 我發現說如果我找遞移關係都透過集合的方式來找的話,實在太麻煩了,每次都要重複寫一樣的東西,於是我想利用矩陣的方法去探討它的遞移關係。如果一個集合裡有n個元素,則我們假設有一個n*n的矩陣,如果$a_{ik} = 1$代表(i,k)在集合裡面,因此現在的問題變成我們必須去找矩陣符合遞移關係的矩陣數量,也就是說如果$a_{ik} = 1$且$a_{kj} = 1$則$a_{ij} = 1$他才會符合條件,其中$1 \leq i, k, j \leq n$
        \item 舉個例子,左邊的矩陣代表\{(1,1),(1,2)\};右邊的矩陣代表\{(1,2),(2,1),(2,2)\}。左邊的矩陣具有遞移關係,而右邊的矩陣不具遞移關係。
    \end{itemize}

    \begin{minipage}{0.4\textwidth}
        \[
        \begin{bmatrix}
        1 & 1 \\
        0 & 0 \\
        \end{bmatrix}
        \]
    \end{minipage}
    \begin{minipage}{0.4\textwidth}
        \[
        \begin{bmatrix}
        0 & 1 \\
        1 & 1 \\
        \end{bmatrix}
        \]
    \end{minipage}
\end{frame}
%%第二部分第二小節
\subsection{製作流程}

%第九頁
\begin{frame}{製作流程}
\framesubtitle{python}
    用python去跑矩陣 ($n \times n$),每個位子都有兩種可能,原理大概是 $i=0$ 到 $i=2^{n^2}-1$,將矩陣的所有 0 和 1 的可能都考慮進去,判斷如果 $a_{ik} = 1$ 且 $a_{kj} = 1$ 則 $a_{ij} = 1$ 的話就將計數器加1,以下有兩張圖分別是我怎麼創造所有可能的矩陣,以及如何用python指令判斷這個矩陣它是否為遞移關係
\end{frame}

%第十頁
\begin{frame}{製作流程}
\framesubtitle{python}
    \begin{figure}[h]
    \centering
    \includegraphics[height=4.5cm]{1.PNG}
    \end{figure}
\end{frame}

\begin{frame}{製作流程}
\framesubtitle{python}
    我們考慮 $n=2$ 的情況。當 $i$ 等於 6 時,它的二進位表示為 110。當 $j$ 遍歷所有的位元位置時,它將檢查 $i$ 的每一個位元。仍然假設 $n$ 等於 2,所以 $j$ 的範圍會是 0 到 3。
    \\
    以下是 $j$ 每一個值對應的結果:
\end{frame}

\begin{frame}{製作流程}
\framesubtitle{python}
    \small
    $j=0$:$(1 \ll 0)$ 等於 1(二進位表示為 0001)。因此,$i \& (1 \ll 0)$ 等於 $110 \& 0001$,等於 0000,這是零的,所以 if 語句不成立,相應的矩陣元素會保持為 0。

    $j=1$:$(1 \ll 1)$ 等於 2(二進位表示為 0010)。因此,$i \& (1 \ll 1)$ 等於 $110 \& 0010$,等於 0010,這是非零的,所以 if 語句成立,相應的矩陣元素會被設置為 1。

    $j=2$:$(1 \ll 2)$ 等於 4(二進位表示為 0100)。因此,$i \& (1 \ll 2)$ 等於 $110 \& 0100$,等於 0100,這是非零的,所以 if 語句成立,相應的矩陣元素會被設置為 1。

    $j=3$:$(1 \ll 3)$ 等於 8(二進位表示為 1000)。然而,$i$ 只有 3 位,因此在這個位置上沒有 1,所以 $i \& (1 \ll 3)$ 的結果將是 0,if 語句不成立,相應的矩陣元素會保持為 0。
\end{frame}

\begin{frame}{製作流程}
\framesubtitle{python}
    所以,當 $i$ 等於 6 時,最終的 $n \times n$ 矩陣如下:

    \[
    \begin{bmatrix}
        0 & 1 \\
        1 & 0
    \end{bmatrix}
    \]
\end{frame}


\begin{frame}{製作流程}
\framesubtitle{python}
    \begin{figure}[h]
    \centering
    \includegraphics[height=5.5cm]{2.PNG}
    \end{figure}
\end{frame}

\begin{frame}{製作流程}
\framesubtitle{python}
    \small
    首先,我們有一個變數 \texttt{is\_transitive},它被初始化為 \texttt{True}。這個變數用於追蹤矩陣是否具有可傳遞性。

    接下來,我們有三個嵌套的迴圈:\texttt{for k in range(n)},\texttt{for i in range(n)},和 \texttt{for j in range(n)}。這些迴圈用於遍歷矩陣的元素。

    在迴圈內部,我們檢查三個條件:

    \begin{itemize}
        \item \texttt{matrix[i][k]}:檢查矩陣的第 $i$ 列和第 $k$ 行是否存在一個非零元素(代表存在一個從頂點 $i$ 到頂點 $k$ 的邊)。
        \item \texttt{matrix[k][j]}:檢查矩陣的第 $k$ 列和第 $j$ 行是否存在一個非零元素(代表存在一個從頂點 $k$ 到頂點 $j$ 的邊)。
        \item \texttt{not matrix[i][j]}:檢查矩陣的第 $i$ 列和第 $j$ 行是否不存在非零元素(代表不存在一個從頂點 $i$ 到頂點 $j$ 的邊)。
    \end{itemize}
\end{frame}

\begin{frame}{製作流程}
\framesubtitle{python}
    \begin{itemize}
    \item \texttt 如果這三個條件都滿足,表示矩陣中存在一個從頂點 $i$ 到頂點 $j$ 的邊,但原本的矩陣中不存在。這違反了可傳遞性的定義,因此將 \texttt{is\_transitive} 設為 \texttt{False}。
    \item 檢查\texttt{is\_transitive} 的值。如果 \texttt{is\_transitive} 仍然為 \texttt{True},表示矩陣是可傳遞性的。在這種情況下,我們將 \texttt{count} 變數加1,表示找到了一個可傳遞性的矩陣。
    \end{itemize}
\end{frame}

\section{成果與討論}
\subsection{遭遇困難}

%第九頁
\begin{frame}{遭遇困難}
\framesubtitle{python}
\begin{itemize}
    \item \texttt 無法找到遞迴關係,因為矩陣只有對角線的位置不會受到影響,而其他位置都會受到影響
    \item \texttt 用python的$i \& (\text{1} \ll j)$的指令創造所有可能的矩陣
    \item \texttt 下圖為第n個元素遞移關係的數量,他在OEIS上已經被登入為一個編號A006905的數列,但我只能跑出前5項,之後就開始卡了,因為時間複雜度太高了
\end{itemize}
\end{frame}

\begin{frame}{遭遇困難}
\framesubtitle{python}
        \begin{figure}[h]
        \includegraphics[totalheight=6cm]{8.PNG}
        \end{figure}
\end{frame}

\subsection{數學背景}

%第九頁
\begin{frame}{數學背景}
\framesubtitle{python}
    \small
    \textbf{\large 關於遞移性}
    \begin{itemize}
        \item \texttt 遞移性檢測:程式碼中的主要目標是檢測給定的關係矩陣是否具有遞移性。它通過遍歷矩陣中的所有元素對 (i, j) 和 (j, k),並檢查是否存在 (i, k)。如果存在 (i, j) 和 (j, k),但不存在 (i, k),則該矩陣不具有遞移性。
        \item \texttt 停止檢測:當發現矩陣中的一對元素 (i, k) 不滿足遞移性時,程式碼會立即停止檢測,並將遞移性標誌設置為 False。這是因為根據遞移性的定義,只需要找到一個不符合的對,就可以確定關係矩陣不具有遞移性,不需要繼續進行檢測。
    \end{itemize}
\end{frame}

\begin{frame}{數學背景}
\framesubtitle{python}
    \small
    \textbf{\large 關於python}
    \begin{itemize}
        \item \texttt 必須理解python的符號概念(i.e., \texttt{\&} 和 \texttt{<<})
        \item \texttt 巢狀迴圈的寫法
        \item \texttt 創造一個n*n的矩陣,matrix = [[0 for \textunderscore in range(n)] for \textunderscore in range(n)]
        \item \texttt 初始化is\textunderscore transitive的概念
    \end{itemize}
\end{frame}

\subsection{成果}

\begin{frame}{成果}
\framesubtitle{python}
    \begin{figure}[h]
    \includegraphics[totalheight=6cm]{9.PNG}
    \end{figure}
\end{frame}

\begin{frame}{成果}
\framesubtitle{python}
    \begin{figure}[h]
    \includegraphics[totalheight=4.5cm]{5.PNG}
    \includegraphics[totalheight=4.5cm]{6.PNG}
    \includegraphics[totalheight=4.5cm]{7.PNG}
    \end{figure}
\end{frame}

\begin{frame}{成果}
\framesubtitle{python}
    \begin{figure}[h]
    \includegraphics[totalheight=2.55cm]{10.PNG}
    \end{figure}
\end{frame}


%%%%
\subsection{參考資料}
\begin{frame}{參考資料}
    \begin{thebibliography}{9}
        \bibitem{g1} Python模組以及指令參考資料. [網址] \url{https://docs.python.org/3/library/stdtypes.html#bitwise-operations-on-integer-types}
        \bibitem{g2} 前17項遞移關係數量、相關證明 [網址] \url{https://www.researchgate.net/publication/352383148_On_the_number_of_transitive_relations_on_a_set}
        \bibitem{g3} 其他關係的補充 [網址] \url{https://www.youtube.com/watch?v=NAdHAiFyU1M}
    \end{thebibliography}
\end{frame}





%%
\end{document}

%動畫示範1
%\begin{frame}{Slide Title \#2}
 %\framesubtitle{Slide subtitle \#1}
	%\begin{itemize}
		%\item Use the \texttt{itemize} environment frequently.
		%\pause
		%\item Use short sentences and phrases.
		%\pause
		%\item In this presentation we use the \textbackslash{}\texttt{pause} macro.
	%\end{itemize}
 %\end{frame}
%動畫示範2:非順序提示
 \begin{frame}{Slide Title \#2}
	\begin{itemize}
		\item <1->You can define the order of appearance.
		\item <3->Like here.
		\item <2->This is the second item to appear.
	\end{itemize}
\end{frame}

%block example
\begin{frame}{Slide Title \#4}
	\begin{example}
		<1->First example. 
	\end{example}
	\begin{example}
		<2->Second example.
	\end{example}
\end{frame}


%table example
\begin{frame}{Slide Title \#5}
	\begin{center}
		Table example \\[12pt]
		\begin{tabular}{c||c|c|c|}
			& \textbf{col 1} & \textbf{col  2} & \textbf{col 3} \\
			\hline
			\hline
			\textbf{row 1} & 11 & 12 & 13 \\
			\hline
			\textbf{row 2} & 21 & 22 & 23 \\
		\end{tabular}
    \end{center}
\end{frame}

\begin{frame}{Slide Title \#6}
	\begin{center}
		Figure example \\[12pt]
		\includegraphics[width=0.35\textwidth,keepaspectratio]{LampFlowchart.png}
		\\
		\footnotesize(source: \textlatin{Wikipedia})
    \end{center}
\end{frame}

\begin{frame}{Slide Title \#7}
	\centering
	Math examples \\[12pt]
	\begin{equation}
        	B'=-\nabla \times E
	\end{equation}
	\begin{equation*}
        	E'=\nabla \times B - 4\pi j
	\end{equation*}
\end{frame}

%alert box
\begin{frame}{Summary}
   	\begin{alertblock}{Attention}
   		\textlatin{This is an important alert}
   	\end{alertblock}
\end{frame}


%references
\begin{frame}{References}
	\begin{thebibliography}{2}
		\beamertemplatebookbibitems
		\bibitem{Author1990}A.\ Author. \newblock\emph{Handbook of Everything}.\newblock
\textlatin{Some Press, \oldstylenums{1990}}.

		\beamertemplatearticlebibitems
		\bibitem{Someone2002}B.\ Author.\newblock On this and that\emph{.}
\newblock\emph{Journal on This and That}. 
\oldstylenums{2}(\oldstylenums{1}):\oldstylenums{50}--\oldstylenums{100}, 
\oldstylenums{2000}.
	\end{thebibliography}
\end{frame}